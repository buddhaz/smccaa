% % @author Shuning Zhang
% % @date 2020-02-04
% \documentclass[a4paper, 11pt]{ctexart}
% \usepackage{amsfonts, amsmath, amssymb}
% \usepackage{color}
% \usepackage{enumerate}
% \usepackage[bottom=2cm, left=2.5cm, right=2.5cm, top=2cm]{geometry}
% \usepackage{multicol}
% \newcommand{\rank}{\mathrm{r}}
% \begin{document}
\begin{enumerate}
    \item % 1
        略.
    \item % 2
        {\heiti 证明}\quad 设齐次线性方程组的基础解析为 $\eta$, 线性无关的向量组为 $\eta'$.
        因为 $\eta \sim \eta'$, 故 $\rank(\eta) = \rank(\eta')$. 又因齐次线性方程组的任一解向量可被 $\eta$ 线性表示, 而 $\eta$ 可被 $\eta'$ 线性表示, 因此齐次线性方程组的任一解向量可被 $\eta'$ 线性表示.
        综上可知 $\eta'$ 是基础解系.
    \item % 3
        {\heiti 证明}\quad 设齐次线性方程组的任意 $n - r$ 个解向量为 $\eta$, 已知 $\eta$ 线性无关, 因此只需证明任一解向量可被 $\eta$ 线性表示即可.
        
        设齐次线性方程组的一个基础解系为 $\eta'$. 将 $\eta$ 和 $\eta'$ 合并成一个新的向量组, 记为 $\eta + \eta'$, 显然这个新的向量组可被 $\eta'$ 线性表示, 而 $\eta'$ 又是线性无关的.
        因此 $\eta'$ 是 $\eta + \eta'$ 极大线性无关部分组. 因此 $\eta + \eta'$ 中任意 $n - r$ 个线性无关的向量也是它的极大线性无关部分组. 因此 $\eta'$ 也可被 $\eta$ 线性表示.
        这样便证明了 $\eta$ 也是一个基础解系.  
    \item % 4
        {\color{red} remained unfinished}{\heiti 证明}\quad 记方程组的系数矩阵为 $A$, 将 $b_1x_1 + \cdots + b_nx_n = 0$ 添加到方程组中, 组成新的方程组, 并记新的方程组的系数矩阵为 $B$.
        显然新的方程组有解, 且 $\rank(A) \leqslant \rank(B)$. 若 $\rank(A) = \rank(B)$, 则说明 $b_1x_1 + \cdots + b_nx_n = 0$ 是一个"无效"方程, 即 $\beta$ 可由 $\alpha_i$ 的极大线性无关部分组表示, 而极大线性无关部分组可由 $\alpha_i$ 线性表示,
        因此 $\beta$ 可被 $\alpha_i$ 线性表示. 若 $\rank(A) < \rank(B)$, 只可能是 $\rank(A) + 1 = \rank(B)$, 即添加进来的新方程组"有效", 即 $\beta$ 通过初等行变换后与 $\alpha_i$ 的极大线性无关部分组组成一个新的极大线性无关部分组.
    \item % 5
        {\heiti 证明}\quad 两个方程组的秩都 $< n/2$, 这说明每个方程组通过初等行变换化为的阶梯型方程组的非零行数 $< n / 2$.
        将这两个阶梯型方程组合并起来, 那么新的方程组的行数必定 $< n$, 再对这个新方程组进行初等行变换, 那么最后的非零行的行数必定 $< n$,
        即这个新方程组的秩 $< n$, 根据定理 3.1 的推论可知, 新方程组必有非零解, 这个非零解正是两组方程的公共解.
    \item % 6
        对方程组的系数矩阵 $A$ 进行初等变化, 即
        \begin{gather*}
            A =
            \begin{bmatrix}
                0 & 1 & 1 & \cdots & 1 \\
                1 & 0 & 1 & \cdots & 1 \\
                \vdots & \vdots & \vdots & \ddots & \vdots \\
                1 & 1 & 1 & \cdots & 0
            \end{bmatrix}
            \rightarrow
            \begin{bmatrix}
                0 & 1 & 1 & \cdots & 1 \\
                1 & -1 & 0 & \cdots & 0 \\
                \vdots & \vdots & \vdots & \ddots & \vdots \\
                1 & 0 & 0 & \cdots & -1
            \end{bmatrix}
            \rightarrow
            \begin{bmatrix}
                n & 1 & 1 & \cdots & 1 \\
                0 & -1 & 0 & \cdots & 0 \\
                \vdots & \vdots & \vdots & \ddots & \vdots \\
                0 & 0 & 0 & \cdots & -1
            \end{bmatrix} \\
            \rightarrow
            \begin{bmatrix}
                n & 0 & 0 & \cdots & 0 \\
                0 & -1 & 0 & \cdots & 0 \\
                \vdots & \vdots & \vdots & \ddots & \vdots \\
                0 & 0 & 0 & \cdots & -1
            \end{bmatrix}
            \rightarrow
            \begin{bmatrix}
                1 & 0 & 0 & \cdots & 0 \\
                0 & 1 & 0 & \cdots & 0 \\
                \vdots & \vdots & \vdots & \ddots & \vdots \\
                0 & 0 & 0 & \cdots & 1
            \end{bmatrix}.
        \end{gather*}
        因此 $\rank(A) = n$, 即方程组只有零解.
    \item % 7
    \item % 8
        略.
    \item % 9
        {\heiti 证明}\quad 设方程组的向量方程为 $\alpha_1x_1 + \alpha_2x_2 + \cdots + \alpha_nx_n = \beta$.
        若 $\beta = 0$, 则 $k_1\eta_1 + k_2\eta_2 + \cdots + k_t\eta_t$ 显然也是方程组的解.
        若 $\beta \neq 0$, 记 $\eta_k^{(i)}\ (1 \leqslant k \leqslant t, 1\leqslant i\leqslant n)$ 为 $\eta_k$ 的第 $i$ 个分量, 对 $k_1\eta_1$, 有
        \begin{align*}
            & \alpha_1(k_1\eta_1^{(1)}) + \alpha_2(k_1\eta_1^{(2)}) + \cdot + \alpha_n(k_1\eta_1^{(n)}) \\
            ={} & k_1\alpha_1\eta_1^{(1)} + k_1\alpha_2\eta_1^{(2)} + \cdot + k_1\alpha_n\eta_1^{(n)} \\
            ={} & k_1(\alpha_1\eta_1^{(1)} + \alpha_2\eta_1^{(2)} + \cdot + \alpha_n\eta_1^{(n)}) \\
            ={} & k_1\beta.    
        \end{align*}
        那么对于 $k_1\eta_1 + k_2\eta_2 + \cdots + k_t\eta_t$, 则有
        \begin{align*}
            & \alpha_1(k_1\eta_1^{(1)} + k_2\eta_2^{(1)} + \cdots + k_t\eta_k^{(1)}) + \cdots + \alpha_n(k_1\eta_1^{(n)} + k_2\eta_2^{(n)} + \cdots + k_t\eta_k^{(n)}) \\
            ={} & k_1(\alpha_1\eta_1^{(1)} + \alpha_2\eta_1^{(2)} + \cdot + \alpha_n\eta_1^{(n)}) + \cdots k_t(\alpha_1\eta_t^{(1)} + \alpha_2\eta_t^{(2)} + \cdot + \alpha_n\eta_t^{(n)}) \\
            ={} & k_1\beta + \cdots + k_t\beta \\
            ={} & (k_1 + \cdots + k_t)\beta \\
            ={} & \beta.
        \end{align*}
        因此 $k_1\eta_1 + k_2\eta_2 + \cdots + k_t\eta_t$ 也是非齐次方程组的解.
    \item % 10
        {\color{red} remained lacked}
        $n$ 个平面对应的对应的线性方程组为
        \[
            \begin{cases}
                A_1x + B_1y + C_1z = -D_1, \\
                A_2x + B_2y + C_2z = -D_2, \\
                \cdots, \\
                A_nx + B_ny + C_nz = -D_n.
            \end{cases}
        \]
        $n$ 平面经过同一点, 即方程组只有唯一解. 根据定理 3.3, 即要求系数矩阵的秩 $=$ 增广矩阵的秩 $= 3$ 即可 .
    \item % 11
        {\heiti 证明} 根据题意, 显然有不等式
        \[
            \rank(A) \leqslant \rank(\overline{A}) \leqslant \rank(B)    
        \]
        成立. 又因 $\rank(A) = \rank(B)$, 因此 $\rank(A) = \rank(\overline{A})$. 根据定理 3.2 即可得知方程组有解.
    \item % 12
        写出线性方程组的增广矩阵, 并作初等变换, 即
        \begin{gather*}
            \begin{bmatrix}
                a & a & \cdots & a & b & b_1 \\
                a & a & \cdots & b & a & b_2 \\
                \vdots & \vdots &  & \vdots & \vdots \\
                a & b & \cdots & a & a & b_{n-1} \\
                b & a & \cdots & a & a & b_n
            \end{bmatrix}
            \rightarrow
            \begin{bmatrix}
                b & a & \cdots & a & a & b_n \\
                a & b & \cdots & a & a & b_{n-1} \\
                \vdots & \vdots &  & \vdots & \vdots \\
                a & a & \cdots & b & a & b_2 \\
                a & a & \cdots & a & b & b_1
            \end{bmatrix} \\
            \rightarrow
            \begin{bmatrix}
                b-a & 0 & \cdots & 0 & a & b_n \\
                0 & b-a & \cdots & 0 & a & b_{n-1} \\
                \vdots & \vdots &  & \vdots & \vdots \\
                0 & 0 & \cdots & b-a & a & b_2 \\
                a-b & a-b & \cdots & a-b & b & b_1
            \end{bmatrix} \\
            \rightarrow
            \begin{bmatrix}
                b-a & 0 & \cdots & 0 & a & b_n \\
                0 & b-a & \cdots & 0 & a & b_{n-1} \\
                \vdots & \vdots &  & \vdots & \vdots \\
                0 & 0 & \cdots & b-a & a & b_2 \\
                0 & 0 & \cdots & 0 & b-(n-1)a & b_1 - \sum_{i=2}^nb_i
            \end{bmatrix}.
        \end{gather*}
        当 $b = (n-1)a$ 且 $b_1 = \sum_{i=2}^nb_i$ 且 $b \ne a$ 时, $\rank(A) = \rank(\overline{A}) = n - 1 < n$, 此时方程组有无穷多组解.
        
        当 $b \ne (n-1)a$ 且 $b_1 \ne \sum_{i=2}^nb_i$ 且 $b \ne a$ 时, $\rank(A) = \rank(\overline{A}) = n$, 此时方程组只有唯一解.
    \item % 13
        {\heiti 证明}\quad 方程组的全部解可写成 $\gamma_0 + l_1\eta_1 + l_2\eta_2 + \cdots + l_s\eta_s$, 令
        \begin{gather*}
            k_0 = 1 - (l_1 + l_2 + \cdots + l_s), \\
            k_1 = l_1, \\
            \cdots, \\
            k_s = l_s. \\
        \end{gather*}
        显然 $k_0 + k_1 + \cdots + k_s = 1$, 而且
        \begin{align*}
            & \gamma_0 + l_1\eta_1 + l_2\eta_2 + \cdots + l_s\eta_s \\
            ={} & (k_0 + k_1 + \cdots + k_s)\gamma_0 + k_1\eta_1 + \cdots + k_s\eta_s \\
            ={} & k_0\gamma_0 + k_1(\gamma_0 + \eta_1) + \cdots + k_s(\gamma_0 + \eta_s) \\
            ={} & k_0\gamma_0 + k_1\gamma_1 + \cdots + k_s\gamma_s.   
        \end{align*}
    \item % 14
        令 $n - s = r$, 这 $s + 1$ 个向量就取第 13 题中的 $s + 1$ 向量, 已知 $\gamma_0, \gamma_1, \cdots, \gamma_s$ 可以表示任意一个解向量, 现在只需证明它们线性无关即可.
        任取 $k_0, k_1, \cdots, k_s \in K$, 使
        \[
            k_0\gamma_0 + k_1\gamma_1 + \cdots + k_s\gamma_s
            =(k_0 + k_1 + \cdots + k_s)\gamma_0 + k_1\eta_1 + \cdots + k_s\eta_s = 0.
        \]
        若 $k_0 + k_1 + \cdots + k_s$ 不等于零, 那么
        \[
            \gamma_0 = -\frac{k_1}{k_0 + \cdots + k_s}\eta_1 - \cdots - \frac{k_s}{k_0 + \cdots + k_s}\eta_s.    
        \] 
        因为 $\gamma_0$ 是非齐次方程组特解, 因此将其带入方程组, 方程组的右边不可能等于 $0$.
        但 $\eta_1, \cdots \eta_s$ 是导出方程 (齐次方程组) 的解, 然么它的任意线性组合代入其中, 右边等于 $0$.
        因此
        \[
            \gamma_0 \ne -\frac{k_1}{k_0 + \cdots + k_s}\eta_1 - \cdots - \frac{k_s}{k_0 + \cdots + k_s}\eta_s,   
        \]
        即 $k_0 + k_1 + \cdots + k_s$ 必定等于 $0$, 则有
        \[
            k_1\eta_1 + \cdots + k_s\eta_s = 0,   
        \]
        $\eta_1, \cdots, \eta_s$ 基础解系, 是线性无关的, 因此 $k_0 = k_1 = k_2 = \cdots = k_s = 0$.
        这样便证明这 $s + 1$ 个向量 $\gamma_0, \gamma_1, \cdots, \gamma_s$ 线性无关.
    \item % 15
        {\color{red} remained lacked}
    \item % 16
        {\heiti 证明}\quad 对方程组的系数矩阵进行初等变换, 即
        \begin{gather*}
            \begin{bmatrix}
                \lambda & a_{12} & \cdots & a_{1(n-1)} & a_{1n} \\
                a_{21} & \lambda & \cdots & a_{2(n-1)} & a_{2n} \\
                \vdots & \vdots & \ddots & \vdots & \vdots \\ 
                a_{(n-1)1} & a_{(n-1)2} & \cdots & \lambda & a_{(n-1)n} \\
                a_{n1} & a_{n2} & \cdots & a_{n(n-1)} & \lambda     
            \end{bmatrix}
        \end{gather*}
\end{enumerate}
% \end{document}
