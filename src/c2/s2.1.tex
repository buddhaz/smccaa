% % @author Shuning Zhang
% % @date 2020-01-30
% \documentclass[a4paper, 11pt]{ctexart}
% \usepackage{amsfonts, amsmath, amssymb}
% \usepackage{enumerate}
% \usepackage[top=2cm, bottom=2cm, left=2.5cm, right=2.5cm]{geometry}
% \usepackage{multicol}
% \begin{document}
\begin{enumerate}
    \item % 1
        略.
    \item % 2
        略.
    \item % 3
        略.
    \item % 4
        {\heiti 证明}\quad 令 $\beta_1 = \alpha_1 + \alpha_2$, $\beta_2 = \alpha_2 + \alpha_3$, $\beta_3 = \alpha_3 + \alpha_1$.
        一方面, $\beta_1$, $\beta_2$, $\beta_3$ 显然可以被 $\alpha_1$, $\alpha_2$, $\alpha_3$ 线性表示.
        另一方面, 有
        \begin{gather*}
            \alpha_1 = \frac{\beta_1 - \beta_2 + \beta_3}{2}, \\
            \alpha_2 = \frac{\beta_1 + \beta_2 - \beta_3}{2}, \\
            \alpha_3 = \frac{-\beta_1 + \beta_2 + \beta_3}{2}.
        \end{gather*}
        因此 $\alpha_1$, $\alpha_2$, $\alpha_3$ 与 $\beta_1$, $\beta_2$, $\beta_3$ 线性等价.
    \item % 5
        {\heiti 证明}\quad 取 $k_1$, $k_2$, $\cdots$, $k_s \in K$, 使
        \begin{align*}
                & k_1\alpha_1 + k_2(\alpha_1 + \alpha_2) + \cdots + k_s(\alpha_1 + \alpha_2 + \cdots + \alpha_s) \\
            ={} & (k_1 + k_2 + \cdots + k_s)\alpha_1 + (k_2 + \cdots + k_s)\alpha_2 + \cdots + k_s\alpha_s \\
            ={} & 0.    
        \end{align*}
        因为 $\alpha_1$, $\alpha_2$, $\cdots$, $\alpha_s$ 线性无关, 所以 $k_1 + k_2 + \cdots + k_s = k_2 + \cdots + k_s = \cdots = k_s = 0$,
        即 $k_1 = k_2 = \cdots = k_s = 0$. 因此 $\alpha_1$, $\alpha_1 + \alpha_2$, $\cdots$, $\alpha_1 + \alpha_2 + \cdots + \alpha_s$ 线性无关.
    \item % 6
        {\heiti 证明}\quad 取 $k_1$, $k_2$, $\cdots$, $k_s$, $k_{s+1} \in K$, 使
        \[
            k_1\alpha_1 + k_2\alpha_2 + \cdots + k_s\alpha_s + k_{s+1}\beta = 0.    
        \]
        若 $k_{s+1} = 0$, 则 $\alpha_1$, $\alpha_2$, $\cdots$, $\alpha_s$, $\beta$ 线性无关, 与题意相悖. 因此 $k_{s+1} \neq 0$, 即
        \[
            \beta = \frac{k_1}{k_{s+1}}\alpha_1 + \frac{k_2}{k_{s+1}}\alpha_2 + \cdots + \frac{k_s}{k_{s+1}}\alpha_s.    
        \]
    \item % 7
        略.
    \item % 8
        略.
    \item % 9
        \begin{enumerate}[(1)]
            \item % 9.1
                {\heiti 证明} 考虑向量方程 $\alpha_1x_1 + \alpha_2x_2 + \cdots + \alpha_mx_m = 0$. 若此向量方程成立, 那么
                \[
                    \alpha_1'x_1 + \alpha_2'x_2 + \cdots + \alpha_m'x_m = 0    
                \] 也成立. 因此若 $\alpha_1'$, $\alpha_2'$, $\cdots$, $\alpha_m'$ 线性无关, 则 $x_1 = \cdots = x_m = 0$,
                进而 $\alpha_1$, $\alpha_2$, $\cdots$, $\alpha_m$ 也线性无关.
            \item % 9.2
                这是 $(1)$ 的逆否命题.
        \end{enumerate}
    \item % 10
        {\heiti 证明} 只需考虑向量方程 $\alpha_1x_1 + \alpha_2x_2 + \cdots + \alpha_mx_m = 0$ 和 $\alpha_1x_1 + \alpha_2x_2 + \cdots + \alpha_mx_m = \beta$. 对于 $(1)$, $(2)$, $(3)$ 这三种变换便是求解线性方程组的初等行变换,
        初等行变换不改变方程的解.
    \item % 11
        略.
    \item % 12
        略.
    \item % 13
        {\heiti 证明}\quad 对 $\alpha_1$, $\alpha_2$, $\cdots$, $\alpha_s$ 中的任一向量 $\alpha_i$, 根据{\heiti 第 6 题}的结论, 可知 $\alpha_i$ 可被 $\alpha_{i_1}$, $\alpha_{i_2}$, $\cdots$, $\alpha_{i_r}$ 线性表示.
        因此 $\alpha_1$, $\alpha_2$, $\cdots$, $\alpha_s$ 可被线性表示, 即 $\alpha_{i_1}$, $\alpha_{i_2}$, $\cdots$, $\alpha_{i_r}$ 是其极大线性无关部分组.
    \item % 14
        {\heiti 证明}\quad 从 $\alpha_1$, $\alpha_2$, $\cdots$, $\alpha_s$ 中任取 $r$ 个线性无关的向量 $\alpha_{i_1}$, $\alpha_{i_2}$, $\cdots$, $\alpha_{i_r}$ 组成一个部分组.
        再从余下的向量中任取个一向量 $\alpha_{i_{r+1}}$ 添加到部分组中. 因为 $\alpha_1$, $\alpha_2$, $\cdots$, $\alpha_s$ 的秩为 $r$, 所以新的部分组 $\alpha_{i_1}$, $\alpha_{i_2}$, $\cdots$, $\alpha_{i_r}$, $\alpha_{i_{r+1}}$ 必定是线性相关的.
        根据{\heiti 第 6 题}的结论, 可知 $\alpha_{i_{r+1}}$ 可被 $\alpha_{i_1}$, $\alpha_{i_2}$, $\cdots$, $\alpha_{i_r}$ 线性表示.
        因此 $\alpha_1$, $\alpha_2$, $\cdots$, $\alpha_s$ 可被 $\alpha_{i_1}$, $\alpha_{i_2}$, $\cdots$, $\alpha_{i_r}$ 线性表示, 即 $\alpha_{i_1}$, $\alpha_{i_2}$, $\cdots$, $\alpha_{i_r}$ 是其极大线性无关部分组.
    \item % 15
        {\heiti 证明}\quad 设 $(\mathrm{\uppercase\expandafter{\romannumeral1}})$ 是 $\alpha_1$, $\alpha_2$, $\cdots$, $\alpha_s$ 的一个极大线性无关部分组, 并记 $\alpha_{i_1}$, $\alpha_{i_2}$, $\cdots$, $\alpha_{i_r}$ 为 $(\mathrm{\uppercase\expandafter{\romannumeral2}})$.
        $(\mathrm{\uppercase\expandafter{\romannumeral1}})$ 与 $(\mathrm{\uppercase\expandafter{\romannumeral2}})$ 显然线性等价, 根据{\heiti 命题 1.5} 的{\heiti 推论 2} 即可知 $(\mathrm{\uppercase\expandafter{\romannumeral1}})$ 与 $(\mathrm{\uppercase\expandafter{\romannumeral2}})$ 的秩相同.
        因此 $(\mathrm{\uppercase\expandafter{\romannumeral2}})$ 的秩为 $r$, 即 $(\mathrm{\uppercase\expandafter{\romannumeral2}})$ 线性无关.
    \item % 16
        {\heiti 证明}\quad 设 $(\mathrm{\uppercase\expandafter{\romannumeral1'}})$ 和 $(\mathrm{\uppercase\expandafter{\romannumeral2'}})$ 分别是 $(\mathrm{\uppercase\expandafter{\romannumeral1}})$ 和 $(\mathrm{\uppercase\expandafter{\romannumeral2}})$ 的极大线性无关部分组, 且 $(\mathrm{\uppercase\expandafter{\romannumeral1'}})$ 的个数为 $r$, $(\mathrm{\uppercase\expandafter{\romannumeral2'}})$ 的个数为 $s$.
        因为 $(\mathrm{\uppercase\expandafter{\romannumeral1'}})$ 可被 $(\mathrm{\uppercase\expandafter{\romannumeral2'}})$ 线性表示, 且 $(\mathrm{\uppercase\expandafter{\romannumeral1'}})$ 线性无关, 所以 $r \leqslant s$. 
    \item % 17
        {\heiti 证明}\quad 已知 $\varepsilon_1$, $\varepsilon_2$, $\cdots$, $\varepsilon_n$ 的秩为 $n$, 再根据{\heiti 第 16 题}的结论, 即可得 $\alpha_1$, $\alpha_2$, $\cdots$, $\alpha_n$ 的秩为 $n$.
        因此 $\alpha_1$, $\alpha_2$, $\cdots$, $\alpha_n$ 线性无关.
    \item % 18
        {\heiti 证明}\quad 先证明充分性. 因为 $n$ 维坐标向量 $\varepsilon_1$, $\varepsilon_2$, $\cdots$, $\varepsilon_n$ 可被 $\alpha_1$, $\alpha_2$, $\cdots$, $\alpha_n$ 线性表示, 根据{\heiti 第 17 题}的结论即可知 $\alpha_1$, $\alpha_2$, $\cdots$, $\alpha_n$ 线性无关.
        
        再证明必要性. 任取 $\alpha_{n+1} \in K^n$ 添加到 $\alpha_1$, $\alpha_2$, $\cdots$, $\alpha_n$ 中组成新向量组 $(\mathrm{\uppercase\expandafter{\romannumeral1}})$. 若 $(\mathrm{\uppercase\expandafter{\romannumeral1}})$ 线性相关, 根据{\heiti 第 6 题}的结论, 可知 $\alpha_{n+1}$ 可被 $\alpha_1$, $\alpha_2$, $\cdots$, $\alpha_n$ 线性表示, 必要性得证.
        若 $(\mathrm{\uppercase\expandafter{\romannumeral1}})$ 线性无关, 已知 $(\mathrm{\uppercase\expandafter{\romannumeral1}})$ 可被 $n$ 维坐标向量 $\varepsilon_1$, $\varepsilon_2$, $\cdots$, $\varepsilon_n$ 线性表示, 但 $(\mathrm{\uppercase\expandafter{\romannumeral1}})$ 的秩 $>$ 后者的秩, 与{\heiti 第 16 题}的结论相悖. 因此 $(\mathrm{\uppercase\expandafter{\romannumeral1}})$ 必定线性相关.
    \item % 19
        提示: 利用筛选法.
    \item % 20
        {\heiti 证明}\quad 记 $\alpha_1$, $\alpha_2$, $\cdots$, $\alpha_s$ 为 $(\mathrm{\uppercase\expandafter{\romannumeral1}})$,
        $\alpha_1$, $\alpha_2$, $\cdots$, $\alpha_s$, $\alpha_{s+1}$, $\cdots$, $\alpha_s$ 为 $(\mathrm{\uppercase\expandafter{\romannumeral2}})$.
        $(\mathrm{\uppercase\expandafter{\romannumeral1}})$ 显然可被 $(\mathrm{\uppercase\expandafter{\romannumeral2}})$ 线性表示, 现只需证明 $(\mathrm{\uppercase\expandafter{\romannumeral2}})$ 可被 $(\mathrm{\uppercase\expandafter{\romannumeral1}})$ 线性表示即可.

        设 $(\mathrm{\uppercase\expandafter{\romannumeral3}})$ 是 $(\mathrm{\uppercase\expandafter{\romannumeral1}})$ 的一个极大线性无关部分组.
        因为 $(\mathrm{\uppercase\expandafter{\romannumeral1}})$ 和 $(\mathrm{\uppercase\expandafter{\romannumeral2}})$ 的秩相同,
        所以 $(\mathrm{\uppercase\expandafter{\romannumeral3}})$ 也是 $(\mathrm{\uppercase\expandafter{\romannumeral2}})$ 的极大线性无关部分组,
        即 $(\mathrm{\uppercase\expandafter{\romannumeral2}})$ 可被 $(\mathrm{\uppercase\expandafter{\romannumeral3}})$ 线性表示.
        又因 $(\mathrm{\uppercase\expandafter{\romannumeral3}})$ 可被 $(\mathrm{\uppercase\expandafter{\romannumeral1}})$ 线性表示,
        所以 $(\mathrm{\uppercase\expandafter{\romannumeral2}})$ 亦可被 $(\mathrm{\uppercase\expandafter{\romannumeral1}})$ 线性表示.
    \item % 21
        {\heiti 证明}\quad 欲证明$\alpha_1$, $\alpha_2$, $\cdots$, $\alpha_r$ 与 $\beta_1$, $\beta_2$, $\cdots$, $\beta_r$ 的秩相同,
        根据{\heiti 第 16 题}的结论, 只需证明 $\alpha_1$, $\alpha_2$, $\cdots$, $\alpha_r$ 与 $\beta_1$, $\beta_2$, $\cdots$, $\beta_r$ 线性等价即可.
        
        一方面, $\beta_1$, $\beta_2$, $\cdots$, $\beta_r$ 显然可被 $\alpha_1$, $\alpha_2$, $\cdots$, $\alpha_r$ 线性表示. 另一方面, 有
        \begin{gather*}
            \alpha_1 = \frac{\beta_1 + \beta_2 + \cdots + \beta_r}{r - 1} - \beta_1, \\
            \alpha_2 = \frac{\beta_1 + \beta_2 + \cdots + \beta_r}{r - 1} - \beta_2, \\
            \cdots, \\
            \alpha_r = \frac{\beta_1 + \beta_2 + \cdots + \beta_r}{r - 1} - \beta_r.    
        \end{gather*}
        因此 $\alpha_1$, $\alpha_2$, $\cdots$, $\alpha_r$ 与 $\beta_1$, $\beta_2$, $\cdots$, $\beta_r$ 线性等价.
    \item % 22
        任取 $k_1$, $k_2 \in K$, 使
        \begin{align*}
                & k_1(\alpha_1 - \alpha_2) + k_2(\alpha_2 - \alpha_3) \\
            ={} & k_1\alpha_1 + (k_2 - k_1)\alpha_2 - k_2\alpha_3 \\
            ={} & 0.
        \end{align*}
        因为 $\alpha_1$, $\alpha_2$, $\alpha_3$ 线性无关, 所以 $k_1 = k_2 - k_1 = -k_2 = 0$, 即 $k_1 = k_2 = 0$.
        因此 $\alpha_1 - \alpha_2$, $\alpha_2 - \alpha_3$ 线性无关. 另外, 有
        \begin{gather*}
            \alpha_1 - \alpha_2 = 1 \cdot (\alpha_1 - \alpha_2) + 0 \cdot (\alpha_2 - \alpha_3), \\
            \alpha_2 - \alpha_3 = 0 \cdot (\alpha_1 - \alpha_2) + 1 \cdot (\alpha_2 - \alpha_3), \\
            \alpha_3 - \alpha_1 = (-1) \cdot (\alpha_1 - \alpha_2) + (-1) \cdot (\alpha_2 - \alpha_3),
        \end{gather*}
        即 $\alpha_1 - \alpha_2$, $\alpha_2 - \alpha_3$, $\alpha_3 - \alpha_1$ 可被 $\alpha_1 - \alpha_2$, $\alpha_2 - \alpha_3$ 线性表示.
        因此 $\alpha_1 - \alpha_2$, $\alpha_2 - \alpha_3$ 是一个极大线性无关部分组.
    \item % 23
        根据第 21 题的结论可知 $\alpha - \alpha_1$, $\cdots$, $\alpha - \alpha_n$ 与 $\alpha_1$, $\cdots$, $\alpha_n$ 同秩, 即 $\alpha - \alpha_1$, $\cdots$, $\alpha - \alpha_n$ 的秩也为 $r$.
        试考虑 $\alpha - \alpha_{i_1}$, $\cdots$, $\alpha - \alpha_{i_r}$ 这组向量. 显然 $\alpha - \alpha_1$, $\cdots$, $\alpha - \alpha_n$ 可被 $\alpha - \alpha_{i_1}$, $\cdots$, $\alpha - \alpha_{i_r}$ 线性表示, 现只需证明其线性无关即可.

        取 $l_1$, $l_2$, $\cdots$, $l_r \in K$, 使
        \begin{align*}
            & l_1(\alpha - \alpha_{i_1}) + l_2(\alpha - \alpha_{i_2}) + \cdots + l_r(\alpha - \alpha_{i_r}) \\
            ={} & l_1((k_1 - 1)\alpha_{i_1} + k_2\alpha_{i_2} + \cdots + k_r\alpha_{i_r}) + \cdots + l_r(k_1\alpha_{i_1} + k_2\alpha_{i_2} + \cdots + (k_r - 1)\alpha_{i_r}) \\
            ={} & (l_1(k_1 - 1) + l_2k_1 + \cdots + l_rk_1)\alpha_{i_1} + \cdots + (l_1k_r + l_2k_r + \cdots + l_r(k_r-1))\alpha_{i_r} \\
            ={} & 0.
        \end{align*}
        因为 $\alpha_{i_1}$, $\cdots$, $\alpha_{i_r}$ 线性无关, 因此
        \begin{gather*}
            l_1(k_1 - 1) + l_2k_1 + \cdots + l_rk_1 = (l_1 + l_2 + \cdots + l_r)k_1 - l_1 = 0 \\
            l_1k_2 + l_2(k_2-1) + \cdots + l_rk_2 = (l_1 + l_2 + \cdots + l_r)k_2 - l_2 = 0 \\
            \cdots \\
            l_1k_r + l_2k_r + \cdots + l_r(k_1-1) = (l_1 + l_2 + \cdots + l_r)k_r - l_r = 0
        \end{gather*}
        将上面的式子相加, 得到
        \begin{align*}
            & (l_1 + l_2 + \cdots + l_r)(k_1 + k_2 + \cdots + k_r) - (l_1 + l_2 + \cdots + l_r) \\
            ={} &  (l_1 + l_2 + \cdots + l_r)(k_1 + k_2 + \cdots + k_r - 1) \\
            ={} & 0.
        \end{align*}
        因为 $k_1 + k_2 + \cdots + k_r \neq 1$, 故 $l_1 + l_2 + \cdots + l_r = 0$, 进而可推出 $l_1 = l_2 = \cdots l_r = 0$ (将 $l_1 + l_2 + \cdots + l_r = 0$ 代回上面的方程组).
    \item % 24
        {\heiti 证明}\quad 用反证法. 假设有不全为零的 $k_1$, $k_2$, $\cdots$, $k_s \in K$ 使得
        \[
            k_1\alpha_1 + k_2\alpha_2 + \cdots + k_s\alpha_s = 0.
        \]
        并假设 $k_i$ 的绝对值最大, 那么
        \[
            \alpha_i = -\frac{k_1}{k_i}\alpha_1 - \frac{k_2}{k_i}\alpha_2 - \cdots - \frac{k_{i-1}}{k_i}\alpha_{i-1} - \frac{k_{i+1}}{k_i}\alpha_{i+1} - \cdots - \frac{k_s}{k_i}\alpha_s.
        \]
        对第 $i$ 个分量, 有
        \begin{align*}
            a_{ii} = -\frac{k_1}{k_i}a_{1i} - \frac{k_2}{k_i}a_{2i} - \cdots - \frac{k_{i-1}}{k_i}a_{(i-1)i} - \frac{k_{i+1}}{k_i}a_{(i+1)i} - \cdots - \frac{k_s}{k_i}a_{si}.
        \end{align*}
        两边取绝对值, 则有
        \begin{align*}
            |a_{ii}| &= |-\frac{k_1}{k_i}a_{1i} - \frac{k_2}{k_i}a_{2i} - \cdots - \frac{k_{i-1}}{k_i}a_{(i-1)i} - \frac{k_{i+1}}{k_i}a_{(i+1)i} - \cdots - \frac{k_s}{k_i}a_{si}| \\
                        &\leqslant |\frac{k_1}{k_i}a_{1i}| + |\frac{k_2}{k_i}a_{2i}| + \cdots + |\frac{k_{i-1}}{k_i}a_{(i-1)i}| + |\frac{k_{i+1}}{k_i}a_{(i+1)i}| + \cdots + |\frac{k_s}{k_i}a_{si}| \\
                        &\leqslant |a_{1i}| + |a_{2i}| + \cdots + |a_{(i-1)i}| + |a_{(i+1)i}| + \cdots + |a_{si}| \\
                        &= \sum_{\substack{l=1\\l \neq i}}^s |a_{li}|.
        \end{align*}
        与题意矛盾, 因此 $k_1, k_2, \cdots, k_s = 0$, 即 $\alpha_1, \alpha_2, \cdots, \alpha_s$ 线性无关.
    \item % 25
        任取 $k_1$, $k_2$, $\cdots$, $k_n \in K$, 使
        \begin{align*}
            & k_1\eta_1 + k_2\eta_2 + \cdots + k_n\eta_n \\
            ={} & (k_1 + k_1a_1, k_1, \cdots, k_1) + \cdots + (k_n, k_n, \cdots, k_n + k_na_n) \\
            ={} & \left(k_1a_1 + \sum_{i=1}^nk_i, k_2a_2 + \sum_{i=1}^nk_i, \cdots, k_na_n + \sum_{i=1}^nk_i\right) \\
            ={} & (0, 0, \cdots, 0) \\
            ={} & 0,
        \end{align*}
        即
        \[
            k_1a_1 = k_2a_2 = \cdots = k_na_n = -\sum_{i=1}^nk_i.    
        \]
        若 $k_1 = k_2 = \cdots = k_n = 0$, 等式显然成立. 若 $k_1$, $k_2$, $\cdots$, $k_n$ 不全为零, 假设 $k_{i_1}$, $k_{i_2}$, $\cdots$, $k_{i_j}$ 不为零, 那么
        \begin{align*}
            \frac{1}{a_{i_1}} + \frac{1}{a_{i_2}} + \cdots + \frac{1}{a_{i_j}} &= -\frac{k_{i_1} + k_{i_2} + \cdots + k_{i_j}}{\displaystyle{\sum_{i=1}^nk_i}} \\
                                                                                &= -\frac{k_{i_1} + k_{i_2} + \cdots + k_{i_j}}{k_{i_1} + k_{i_2} + \cdots + k_{i_j}} \\
                                                                                &= -1,
        \end{align*}
        与题意矛盾. 因此 $k_1 = k_2 = \cdots = k_n = 0$, 即 $\eta_1$, $\eta_2$, $\cdots$, $\eta_n$ 线性无关, 亦即 $\eta_1$, $\eta_2$, $\cdots$, $\eta_n$ 的秩为 $n$.
\end{enumerate}
% \end{document}
