% @author Shuning Zhang
% @date 2020-01-30
\documentclass[a4paper, 11pt]{ctexart}
\usepackage{amsfonts, amsmath, amssymb}
\usepackage{enumerate}
\usepackage[top=2cm, bottom=2cm, left=2.5cm, right=2.5cm]{geometry}
\usepackage{multicol}
\begin{document}
    \begin{enumerate}
        \item % 1
        \item % 2
        \item % 3
        \item % 4
            {\heiti 证明}\quad 令 $\beta_1 = \alpha_1 + \alpha_2$, $\beta_2 = \alpha_2 + \alpha_3$, $\beta_3 = \alpha_3 + \alpha_1$.
            一方面, $\beta_1$, $\beta_2$, $\beta_3$ 显然可以被 $\alpha_1$, $\alpha_2$, $\alpha_3$ 线性表示.
            另一方面, 有
            \begin{gather*}
                \alpha_1 = \frac{\beta_1 - \beta_2 + \beta_3}{2}, \\
                \alpha_2 = \frac{\beta_1 + \beta_2 - \beta_3}{2}, \\
                \alpha_3 = \frac{-\beta_1 + \beta_2 + \beta_3}{2}.
            \end{gather*}
            因此 $\alpha_1$, $\alpha_2$, $\alpha_3$ 与 $\beta_1$, $\beta_2$, $\beta_3$ 线性等价.
        \item % 5
            {\heiti 证明}\quad 取 $k_1$, $k_2$, $\cdots$, $k_s \in K$, 使
            \begin{align*}
                    & k_1\alpha_1 + k_2(\alpha_1 + \alpha_2) + \cdots + k_s(\alpha_1 + \alpha_2 + \cdots + \alpha_s) \\
                ={} & (k_1 + k_2 + \cdots + k_s)\alpha_1 + (k_2 + \cdots + k_s)\alpha_2 + \cdots + k_s\alpha_s \\
                ={} & 0.    
            \end{align*}
            因为 $\alpha_1$, $\alpha_2$, $\cdots$, $\alpha_s$ 线性无关, 所以 $k_1 + k_2 + \cdots + k_s = k_2 + \cdots + k_s = \cdots = k_s = 0$,
            即 $k_1 = k_2 = \cdots = k_s = 0$. 因此 $\alpha_1$, $\alpha_1 + \alpha_2$, $\cdots$, $\alpha_1 + \alpha_2 + \cdots + \alpha_s$ 线性无关.
        \item % 6
            {\heiti 证明}\quad 取 $k_1$, $k_2$, $\cdots$, $k_s$, $k_{s+1} \in K$, 使
            \[
                k_1\alpha_1 + k_2\alpha_2 + \cdots + k_s\alpha_s + k_{s+1}\beta = 0,    
            \]
            若 $k_{s+1} = 0$, 则 $\alpha_1$, $\alpha_2$, $\cdots$, $\alpha_s$, $\beta$ 线性无关, 与题意相悖. 因此 $k_{s+1} \neq 0$, 即
            \[
                \beta = \frac{k_1}{k_{s+1}}\alpha_1 + \frac{k_2}{k_{s+1}}\alpha_2 + \cdots + \frac{k_s}{k_{s+1}}\alpha_s.    
            \]
        \item % 7
            略.
        \item % 8
            略.
        \item % 9
        \item % 10
        \item % 11
            略.
        \item % 12
        \item % 13
            {\heiti 证明}\quad 对 $\alpha_1$, $\alpha_2$, $\cdots$, $\alpha_s$ 中的任一向量 $\alpha_i$, 根据{\heiti 第 6 题}的结论, 可知 $\alpha_i$ 可被 $\alpha_{i_1}$, $\alpha_{i_2}$, $\cdots$, $\alpha_{i_r}$ 线性表示.
            因此 $\alpha_1$, $\alpha_2$, $\cdots$, $\alpha_s$ 可被线性表示, 即 $\alpha_{i_1}$, $\alpha_{i_2}$, $\cdots$, $\alpha_{i_r}$ 是其极大线性无关部分组.
        \item % 14
            {\heiti 证明}\quad 从 $\alpha_1$, $\alpha_2$, $\cdots$, $\alpha_s$ 中任取 $r$ 个线性无关的向量 $\alpha_{i_1}$, $\alpha_{i_2}$, $\cdots$, $\alpha_{i_r}$ 组成一个部分组.
            再从余下的向量中任取个一向量 $\alpha_{i_{r+1}}$ 添加到部分组中. 因为 $\alpha_1$, $\alpha_2$, $\cdots$, $\alpha_s$ 的秩为 $r$, 所以新的部分组 $\alpha_{i_1}$, $\alpha_{i_2}$, $\cdots$, $\alpha_{i_r}$, $\alpha_{i_{r+1}}$ 必定是线性相关的.
            根据{\heiti 第 6 题}的结论, 可知 $\alpha_{i_{r+1}}$ 可被 $\alpha_{i_1}$, $\alpha_{i_2}$, $\cdots$, $\alpha_{i_r}$ 线性表示.
            因此 $\alpha_1$, $\alpha_2$, $\cdots$, $\alpha_s$ 可被 $\alpha_{i_1}$, $\alpha_{i_2}$, $\cdots$, $\alpha_{i_r}$ 线性表示, 即 $\alpha_{i_1}$, $\alpha_{i_2}$, $\cdots$, $\alpha_{i_r}$ 是其极大线性无关部分组.
        \item % 15
        \item % 16
            {\heiti 证明}\quad 设 $(\mathrm{\uppercase\expandafter{\romannumeral1'}})$ 和 $(\mathrm{\uppercase\expandafter{\romannumeral2'}})$ 分别是 $(\mathrm{\uppercase\expandafter{\romannumeral1}})$ 和 $(\mathrm{\uppercase\expandafter{\romannumeral2}})$ 的极大线性无关部分组, 且 $(\mathrm{\uppercase\expandafter{\romannumeral1'}})$ 的个数为 $r$, $(\mathrm{\uppercase\expandafter{\romannumeral2'}})$ 的个数为 $s$.
            因为 $(\mathrm{\uppercase\expandafter{\romannumeral1'}})$ 可被 $(\mathrm{\uppercase\expandafter{\romannumeral2'}})$ 线性表示, 且 $(\mathrm{\uppercase\expandafter{\romannumeral1'}})$ 线性无关, 所以 $r \leqslant s$. 
        \item % 17
            {\heiti 证明}\quad 已知 $\varepsilon_1$, $\varepsilon_2$, $\cdots$, $\varepsilon_n$ 的秩为 $n$, 再根据{\heiti 第 16 题}的结论, 即可得 $\alpha_1$, $\alpha_2$, $\cdots$, $\alpha_n$ 的秩为 $n$.
            因此 $\alpha_1$, $\alpha_2$, $\cdots$, $\alpha_n$ 线性无关.
        \item % 18
            {\heiti 证明}\quad 先证明充分性. 因为 $n$ 维坐标向量 $\varepsilon_1$, $\varepsilon_2$, $\cdots$, $\varepsilon_n$ 可被 $\alpha_1$, $\alpha_2$, $\cdots$, $\alpha_n$ 线性表示, 根据{\heiti 第 17 题}的结论即可知 $\alpha_1$, $\alpha_2$, $\cdots$, $\alpha_n$ 线性无关.
            
            再证明必要性. 任取 $\alpha_{n+1} \in K^n$ 添加到 $\alpha_1$, $\alpha_2$, $\cdots$, $\alpha_n$ 中组成新向量组 $(\mathrm{\uppercase\expandafter{\romannumeral1}})$. 若 $(\mathrm{\uppercase\expandafter{\romannumeral1}})$ 线性相关, 根据{\heiti 第 6 题}的结论, 可知 $\alpha_{n+1}$ 可被 $\alpha_1$, $\alpha_2$, $\cdots$, $\alpha_n$ 线性表示, 必要性得证.
            若 $(\mathrm{\uppercase\expandafter{\romannumeral1}})$ 线性无关, 已知 $(\mathrm{\uppercase\expandafter{\romannumeral1}})$ 可被 $n$ 维坐标向量 $\varepsilon_1$, $\varepsilon_2$, $\cdots$, $\varepsilon_n$ 线性表示, 但 $(\mathrm{\uppercase\expandafter{\romannumeral1}})$ 的秩 $>$ 后者的秩, 与{\heiti 第 16 题}的结论相悖. 因此 $(\mathrm{\uppercase\expandafter{\romannumeral1}})$ 必定线性相关.
        \item % 19
            提示: 利用筛选法.
        \item % 20
            {\heiti 证明}\quad 记 $\alpha_1$, $\alpha_2$, $\cdots$, $\alpha_s$ 为 $(\mathrm{\uppercase\expandafter{\romannumeral1}})$,
            $\alpha_1$, $\alpha_2$, $\cdots$, $\alpha_s$, $\alpha_{s+1}$, $\cdots$, $\alpha_s$ 为 $(\mathrm{\uppercase\expandafter{\romannumeral2}})$.
            $(\mathrm{\uppercase\expandafter{\romannumeral1}})$ 显然可被 $(\mathrm{\uppercase\expandafter{\romannumeral2}})$ 线性表示, 现只需证明 $(\mathrm{\uppercase\expandafter{\romannumeral2}})$ 可被 $(\mathrm{\uppercase\expandafter{\romannumeral1}})$ 线性表示即可.

            设 $(\mathrm{\uppercase\expandafter{\romannumeral3}})$ 是 $(\mathrm{\uppercase\expandafter{\romannumeral1}})$ 的一个极大线性无关部分组.
            因为 $(\mathrm{\uppercase\expandafter{\romannumeral1}})$ 和 $(\mathrm{\uppercase\expandafter{\romannumeral2}})$ 的秩相同,
            所以 $(\mathrm{\uppercase\expandafter{\romannumeral3}})$ 也是 $(\mathrm{\uppercase\expandafter{\romannumeral2}})$ 的极大线性无关部分组,
            即 $(\mathrm{\uppercase\expandafter{\romannumeral2}})$ 可被 $(\mathrm{\uppercase\expandafter{\romannumeral3}})$ 线性表示.
            又因 $(\mathrm{\uppercase\expandafter{\romannumeral3}})$ 可被 $(\mathrm{\uppercase\expandafter{\romannumeral1}})$ 线性表示,
            所以 $(\mathrm{\uppercase\expandafter{\romannumeral2}})$ 亦可被 $(\mathrm{\uppercase\expandafter{\romannumeral1}})$ 线性表示.
        \item % 21
            {\heiti 证明}\quad 欲证明$\alpha_1$, $\alpha_2$, $\cdots$, $\alpha_r$ 与 $\beta_1$, $\beta_2$, $\cdots$, $\beta_r$ 的秩相同,
            根据{\heiti 第 16 题}的结论, 只需证明 $\alpha_1$, $\alpha_2$, $\cdots$, $\alpha_r$ 与 $\beta_1$, $\beta_2$, $\cdots$, $\beta_r$ 线性等价即可.
            
            一方面, $\beta_1$, $\beta_2$, $\cdots$, $\beta_r$ 显然可被 $\alpha_1$, $\alpha_2$, $\cdots$, $\alpha_r$ 线性表示. 另一方面, 有
            \begin{gather*}
                \alpha_1 = \frac{\beta_1 + \beta_2 + \cdots + \beta_r}{r - 1} - \beta_1, \\
                \alpha_2 = \frac{\beta_1 + \beta_2 + \cdots + \beta_r}{r - 1} - \beta_2, \\
                \cdots, \\
                \alpha_r = \frac{\beta_1 + \beta_2 + \cdots + \beta_r}{r - 1} - \beta_r.    
            \end{gather*}
            因此 $\alpha_1$, $\alpha_2$, $\cdots$, $\alpha_r$ 与 $\beta_1$, $\beta_2$, $\cdots$, $\beta_r$ 线性等价.
        \item % 22
            任取 $k_1$, $k_2 \in K$, 使
            \begin{align*}
                    & k_1(\alpha_1 - \alpha_2) + k_2(\alpha_2 - \alpha_3) \\
                ={} & k_1\alpha_1 + (k_2 - k_1)\alpha_2 - k_2\alpha_3 \\
                ={} & 0.
            \end{align*}
            因为 $\alpha_1$, $\alpha_2$, $\alpha_3$ 线性无关, 所以 $k_1 = k_2 - k_1 = -k_2 = 0$, 即 $k_1 = k_2 = 0$.
            因此 $\alpha_1 - \alpha_2$, $\alpha_2 - \alpha_3$ 线性无关. 另外, 有
            \begin{gather*}
                \alpha_1 - \alpha_2 = 1 \cdot (\alpha_1 - \alpha_2) + 0 \cdot (\alpha_2 - \alpha_3), \\
                \alpha_2 - \alpha_3 = 0 \cdot (\alpha_1 - \alpha_2) + 1 \cdot (\alpha_2 - \alpha_3), \\
                \alpha_3 - \alpha_1 = (-1) \cdot (\alpha_1 - \alpha_2) + (-1) \cdot (\alpha_2 - \alpha_3),
            \end{gather*}
            即 $\alpha_1 - \alpha_2$, $\alpha_2 - \alpha_3$, $\alpha_3 - \alpha_1$ 可被 $\alpha_1 - \alpha_2$, $\alpha_2 - \alpha_3$ 线性表示.
            因此 $\alpha_1 - \alpha_2$, $\alpha_2 - \alpha_3$ 是一个极大线性无关部分组.
        \item % 23
        \item % 24
        \item % 25
            任取 $k_1$, $k_2$, $\cdots$, $k_n \in K$, 使
            \begin{align*}
                & k_1\eta_1 + k_2\eta_2 + \cdots + k_n\eta_n \\
                ={} & (k_1 + k_1a_1, k_1, \cdots, k_1) + \cdots + (k_n, k_n, \cdots, k_n + k_na_n) \\
                ={} & \left(k_1a_1 + \sum_{i=1}^nk_i, k_2a_2 + \sum_{i=1}^nk_i, \cdots, k_na_n + \sum_{i=1}^nk_i\right) \\
                ={} & (0, 0, \cdots, 0) \\
                ={} & 0,
            \end{align*}
            即
            \[
                k_1a_1 = k_2a_2 = \cdots = k_na_n = -\sum_{i=1}^nk_i.    
            \]
            若 $k_1 = k_2 = \cdots = k_n = 0$, 等式显然成立. 若 $k_1$, $k_2$, $\cdots$, $k_n$ 不全为零, 假设 $k_{i_1}$, $k_{i_2}$, $\cdots$, $k_{i_j}$ 不为零, 那么
            \begin{align*}
                \frac{1}{a_{i_1}} + \frac{1}{a_{i_2}} + \cdots + \frac{1}{a_{i_j}} &= -\frac{k_{i_1} + k_{i_2} + \cdots + k_{i_j}}{\displaystyle{\sum_{i=1}^nk_i}} \\
                                                                                   &= -\frac{k_{i_1} + k_{i_2} + \cdots + k_{i_j}}{k_{i_1} + k_{i_2} + \cdots + k_{i_j}} \\
                                                                                   &= -1,
            \end{align*}
            与题意矛盾. 因此 $k_1 = k_2 = \cdots = k_n = 0$, 即 $\eta_1$, $\eta_2$, $\cdots$, $\eta_n$ 线性无关, 亦即 $\eta_1$, $\eta_2$, $\cdots$, $\eta_n$ 的秩为 $n$.
    \end{enumerate}
\end{document}
