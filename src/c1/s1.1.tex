\documentclass[a4paper, 11pt]{ctexart}
\usepackage{amsfonts, amsmath, amssymb}
\usepackage{enumerate}
\usepackage[top=2cm, bottom=2cm, left=2.5cm, right=2.5cm]{geometry}
\usepackage{multicol}
\begin{document}
    \begin{enumerate}
        \item % 1
            \begin{enumerate}[(1)]
                \item % 1.1
                    若 $a$, $b$, $c$, $d \in \mathbb{Q}$, 且 $c$, $d \neq 0$.  一方面, 有
                    \[
                        a + b\sqrt{2} \pm (c + d\sqrt{2}) = (a \pm c) + (b \pm d)\sqrt{2} \in  \mathbb{Q}(\sqrt{2}).    
                    \]
                    另一方面, 有
                    \begin{gather*}
                        (a + b\sqrt{2}) * (c + d\sqrt{2}) = (ac + 2bd) + (ad + bc)\sqrt{2} \in \mathbb{Q}(\sqrt{2}), \\
                        \frac{a + b\sqrt{2}}{c + d\sqrt{2}} = \frac{(a + b\sqrt{2})(c - d\sqrt{2})}{c^2 - 2d^2} = \frac{ac - 2bd}{c^2 - 2d^2} + \frac{bc - ad}{c^2 - 2d^2}\sqrt{2} \in \mathbb{Q}\sqrt{2}.    
                    \end{gather*}
                    因此 $\mathbb{Q}(\sqrt{2})$ 是一个数域. 
                \item % 1.2
                    略.
                \item % 1.3
                    略.
            \end{enumerate}
        \item % 2
            \begin{multicols}{3}
                \begin{enumerate}[(1)]
                    \item % 2.1
                        单射;
                    \item % 2.2
                        满射;
                    \item % 2.3
                        单射.
                \end{enumerate}
            \end{multicols}
        \item % 3
        \item % 4
            \begin{enumerate}[(1)]
                \item % 4.1
                    先证明 $A \cap (B \cup C) \subset (A \cap B) \cup (A \cap C)$, 再证明 $(A \cap B) \cup (A \cap C) \subset A \cap (B \cup C)$ 即可.
                \item % 4.2
                    略.
            \end{enumerate}
        \item % 5
        \item % 6
            {\heiti 证明}\quad 先证明 $(0, 1)$ 里的有理数与正整数集之间有一个一一对应. 可以将 $(0, 1)$ 里的有理数排列如下:
            \begin{gather*}
                \frac12, \\
                \frac13, \frac23, \\
                \frac14, \frac24, \frac34, \\
                \frac15, \frac25, \frac35, \frac45, \\
                \cdots
            \end{gather*}
            这样便将 $(0, 1)$ 中所有的有理数排列出来,并且不会有遗漏. 再将上面的有理数排成一排并剔除重复的数, 则所得的一列数便与正整数集合之间建立了一个一一对应.
        
            对于其它区间 $(0 + n, 1 + n)\ (n \in \mathbb{Z})$, 采用同样的方法 (排列的时候对每个数加 $n$) 即可. 至于 $\{0, \pm1, \pm2, \cdots\}$, 它们本身就与正整数集是一一对应的.
            这样便建立了 $\mathbb{Q}$ 与正整数集之间的一个一一对应.
        \item % 7
        \item % 8
        \item % 9
        \item % 10
            \begin{enumerate}[(1)]
                \item % 10.1
                    $\dfrac{n(n+1)}{2}$, $\dfrac{n(n+1)(2n+1)}{6}$, $\dfrac{n((n+1)(2n+10) + 12)}{6}$.
                \item % 10.2
                    \[
                        \sum_{i=1}^n(-1)^i =
                        \begin{cases}
                            0, & \text{若 $n$ 为偶数}, \\
                            1, & \text{若 $n$ 为奇数}.
                        \end{cases}    
                    \]
                    \[
                        \sum_{i=1}^n(-1)^ii =
                        \begin{cases}
                            \dfrac{n}{2}, & \text{若 $n$ 为偶数}, \\
                            -\dfrac{n+1}{2}, & \text{若 $n$ 为奇数}.
                        \end{cases}    
                    \]
            \end{enumerate}
        \item % 11
            {\heiti 证明}\quad \begin{align*}
                \sum_{i=1}^n\frac{1}{i(i+1)} &= \sum_{i=1}^n\left(\frac1i - \frac{1}{i+1}\right) \\
                                             &= \sum_{i=1}^n\frac1i - \sum_{i=1}^n\frac{1}{i+1} \\
                                             &= \left(1 + \frac12 + \cdots + \frac1n\right) - \left(\frac12 + \frac13 + \cdots + \frac{1}{n+1}\right) \\
                                             &= 1 - \frac{1}{n+1}.
            \end{align*}
        \item % 12
        \item % 13
        \item % 14
        \item % 15
        \item % 16
    \end{enumerate}
\end{document}
