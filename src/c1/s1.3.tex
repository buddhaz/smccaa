\documentclass[a4paper, 11pt]{ctexart}
\usepackage{amsfonts, amsmath, amssymb}
\usepackage{enumerate}
\usepackage[top=2cm, bottom=2cm, left=2.5cm, right=2.5cm]{geometry}
\usepackage{multicol}
\begin{document}
    \begin{enumerate}
        \item % 1
            \begin{enumerate}[(1)]
                \item % 1.1
                    $x_1 = 0$, $x_2 = 2$, $x_3 = \dfrac53$, $x_4 = -\dfrac43$.
                \item % 1.2
                    $x_1 = -\dfrac{x_5}{2}$, $x_2 = -\dfrac{x_5}{2} - 1$, $x_3 = 0$, $x_4 = -\dfrac{x_5}{2} - 1$.
                \item % 1.3
                    无解.
                \item % 1.4
                    $x_1 = -8$, $x_2 = 3$, $x_3 = 6$, $x_4 = 0$.
                \item % 1.5
                    $x_1 = \dfrac{3}{17}x_3 - \dfrac{13}{17}x_4$, $x_2 = \dfrac{19}{17}x_3 - \dfrac{20}{17}x_4$.
                \item % 1.6
                    无解.
                \item % 1.7
                    $x_1 = \dfrac56x_4 + \dfrac16$, $x_2 = -\dfrac76x_4 + \dfrac16$, $x_3 = \dfrac56x_4 + \dfrac16$.
            \end{enumerate}
        \item % 2
            {\heiti 证明}\quad 利用矩阵消元法, 则有
            \[
                \begin{bmatrix}
                    a_{11} & a_{12} \\
                    a_{21} & a_{22}
                \end{bmatrix}
                \rightarrow
                \begin{bmatrix}
                    -a_{21} & -\dfrac{a_{12}a_{21}}{a_{11}} \\
                    a_{21} & a_{22}
                \end{bmatrix}
                \rightarrow
                \begin{bmatrix}
                    -a_{21} & -\dfrac{a_{12}a_{21}}{a_{11}} \\
                    0 & \dfrac{a_{11}a_{22} - a_{12}a_{21}}{a_{11}}
                \end{bmatrix}. 
            \]
            因此 $a_{11}a_{22} - a_{12}a_{21} = 0$.
        \item % 3
            \begin{enumerate}[(1)]
                \item % 3.1
                \item % 3.2
                \item % 3.3
                    有.
                \item % 3.4
            \end{enumerate}
        \item % 4
            $a = \dfrac13$. $x_1 = 3x_3$, $x_2 = -7x_3$.
        \item % 5
            略.
        \item % 6
            {\heiti 证明}\quad 利用矩阵消元法, 则有
            \begin{gather*}
                \begin{bmatrix}
                    1 & -1 & 0 & 0 & 0 & a_1 \\
                    0 & 1 & -1 & 0 & 0 & a_2 \\
                    0 & 0 & 1 & -1 & 0 & a_3 \\ 
                    0 & 0 & 0 & 1 & -1 & a_4 \\
                    -1 & 0 & 0 & 0 & 1 & a_5
                \end{bmatrix} \rightarrow
                \begin{bmatrix}
                    1 & -1 & 0 & 0 & 0 & a_1 \\
                    0 & 1 & -1 & 0 & 0 & a_2 \\
                    0 & 0 & 1 & -1 & 0 & a_3 \\ 
                    0 & 0 & 0 & 1 & -1 & a_4 \\
                    0 & -1 & 0 & 0 & 1 & a_1 + a_5
                \end{bmatrix} \\
                \rightarrow \begin{bmatrix}
                    1 & -1 & 0 & 0 & 0 & a_1 \\
                    0 & 0 & -1 & 0 & 1 & a_1 + a_2 + a_5 \\
                    0 & 0 & 1 & -1 & 0 & a_3 \\ 
                    0 & 0 & 0 & 1 & -1 & a_4 \\
                    0 & -1 & 0 & 0 & 1 & a_1 + a_5
                \end{bmatrix}
                \rightarrow \begin{bmatrix}
                    1 & -1 & 0 & 0 & 0 & a_1 \\
                    0 & 0 & -1 & 0 & 1 & a_1 + a_2 + a_5 \\
                    0 & 0 & 0 & -1 & 1 & a_1 + a_2 + a_3 + a_5 \\ 
                    0 & 0 & 0 & 1 & -1 & a_4 \\
                    0 & -1 & 0 & 0 & 1 & a_1 + a_5
                \end{bmatrix} \\
                \rightarrow \begin{bmatrix}
                    1 & -1 & 0 & 0 & 0 & a_1 \\
                    0 & 0 & -1 & 0 & 1 & a_1 + a_2 + a_5 \\
                    0 & 0 & 0 & -1 & 1 & a_1 + a_2 + a_3 + a_5 \\ 
                    0 & 0 & 0 & 0 & 0 & a_1 + a_2 + a_3 + a_4 + a_5\\
                    0 & -1 & 0 & 0 & 1 & a_1 + a_5
                \end{bmatrix}.
            \end{gather*}
            因此 $\sum\limits_{i=1}^5a_i = 0$. 进一步, 解得 $x_1 = x_5 - a_5$, $x_2 = x_5 - (a_1 + a_5)$, $x_3 = x_5 - (a_1 + a_2 + a_5)$, $x_4 = x_5 - (a_1 + a_2 + a_3 + a_5)$.
        \item % 7
            利用矩阵消元法, 则有
            \begin{gather*}
                \begin{bmatrix}
                    1 & 0 & 0 & \cdots & 0 & 0 & 1 \\
                    1 & 1 & 0 & \cdots & 0 & 0 & 0 \\
                    0 & 1 & 1 & \cdots & 0 & 0 & 0 \\
                    \vdots & \vdots & \vdots & \ddots & \vdots & \vdots & \vdots \\
                    0 & 0 & 0 & \cdots & 1 & 0 & 0 \\
                    0 & 0 & 0 & \cdots & 1 & 1 & 0 \\
                    0 & 0 & 0 & \cdots & 0 & 1 & 1
                \end{bmatrix}
                \rightarrow
                \begin{bmatrix}
                    1 & 0 & 0 & \cdots & 0 & 0 & 1 \\
                    0 & 1 & 0 & \cdots & 0 & 0 & -1 \\
                    0 & 0 & 1 & \cdots & 0 & 0 & 1 \\
                    \vdots & \vdots & \vdots & \ddots & \vdots & \vdots & \vdots \\
                    0 & 0 & 0 & \cdots & 1 & 0 & (-1)^{n-3} \\
                    0 & 0 & 0 & \cdots & 0 & 1 & (-1)^{n-2} \\
                    0 & 0 & 0 & \cdots & 0 & 0 & 1 - (-1)^{n-2}
                \end{bmatrix}.
            \end{gather*}
            考察 $x_n$ 前的系数 $1 - (-1)^{n-2}$. 当 $n$ 为奇数时, 解得 $x_1 = x_2 = \cdots = x_n = 0$.
            当 $n$ 为偶数时, 解得 $x_1 = x_n$, $x_2 = -x_n$, $\cdots$, $x_{n-1} = (-1)^{n-1}x_n$.
        \item % 8
        \item % 9
        \item % 10
    \end{enumerate}
\end{document}
